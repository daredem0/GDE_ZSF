%%%%%Präambel%%%%%

\documentclass[12pt,a4paper]{article}%Schriftgröße, Papierformat einstellen
%\documentclass{scrbook}
\usepackage[top=30mm,bottom=30mm]{geometry}
\usepackage{lipsum}
%Pakete laden zur deutschen Rechtschreibung und für Umlaute
\usepackage[T1]{fontenc}
\usepackage[ngerman]{babel}
\usepackage[utf8]{inputenc} %für Windows, Linux
%\usepackage[applemac]{inputenc} %für Mac
%\usepackage{xcolor}
\usepackage[dvipsnames]{xcolor}
\usepackage{cancel}
\usepackage{titlesec}
\usepackage{cite}
\usepackage{filecontents}
\usepackage{harvard}
\let\harvardleftorig\harvardleft
%\usepackage[round]{natbib}
%\usepackage{hyperref}

%Inhaltsverzeichnis mit Links erstellen
\usepackage[colorlinks,
pdfpagelabels,
pdfstartview = FitH,
bookmarksopen = true,
bookmarksnumbered = true,
linkcolor = black,
plainpages = false,
hypertexnames = false,
citecolor = black] {hyperref}

%Pakete laden zu mathematischen Symbolen etc.
\usepackage{calc} 
\usepackage{amsmath,amssymb,amsthm}
\usepackage{scrpage2}
\pagestyle{scrheadings}
\clearscrheadfoot
\automark[chapter]{section}
\ofoot{\pagemark}
\ifoot{Florian Leuze}
\chead{\headmark}
\setfootsepline{1pt}
\setheadsepline{1pt}
%\setheadsepline[\textwidth+20pt]{0.5pt}

% Umgebungen für Definitionen, Sätze, usw.
\newtheorem{satz}{Satz}[section]
\newtheorem{definition}[satz]{Definition}     
\newtheorem{lemma}[satz]{Lemma}	
% Es werden Sätze, Definitionen etc innerhalb einer Section mit
% 1.1, 1.2 etc durchnummeriert, ebenso die Gleichungen mit (1.1), (1.2) ..                  
\numberwithin{equation}{section}

\setcounter{secnumdepth}{4}

\titleformat{\paragraph}
{\normalfont\normalsize\bfseries}{\theparagraph}{1em}{}
\titlespacing*{\paragraph}
{0pt}{3.25ex plus 1ex minus .2ex}{1.5ex plus .2ex}

%neue Befehle definieren
\newcommand{\R}{\mathbb{R}} %zB \R als Abkürzung für das Symbol der reellen Zahlen
\newcommand{\subsubsubsection}{\paragraph}
\newcommand\citevgl
{\def\harvardleft{(vgl.\ \global\let\harvardleft\harvardleftorig}%
 \cite
}
\newcommand\citeVgl
{\def\harvardleft{(Vgl.\ \global\let\harvardleft\harvardleftorig}%
 \cite
}

%Makros
%Makro Color
%#1 Text
\def\colBord#1{\begingroup\color{Fuchsia}{#1}\endgroup}
\def\colRed#1{\begingroup\color{Red}{#1}\endgroup}
\def\colGreen#1{\begingroup\color{LimeGreen}{#1}\endgroup}
\def\colBlue#1{\begingroup\color{NavyBlue}{#1}\endgroup}

\def\usGreen#1#2{\underset{\colGreen{#1}}{#2}}
\def\usBord#1#2{\underset{\colBord{#1}}{#2}}

% Auf der Seite http://detexify.kirelabs.org/classify.html können Sie mathematische Symbole, Pfeile usw per Maus eingeben und bekommen den Latex-Befehl dafür angezeigt.
% detexify gibt es auch als App...



%jetzt beginnt das eigentliche Dokument
\begin{document}
\bibliographystyle{agsm}

\author{}
\title{\underline{HM1-2 Zusammenfassung}}
\date{}

\maketitle % erzeugt den Kopf
\newpage
\section{\underline{Inhalt}}
\tableofcontents

\subsection{Versionierung}
\begin{tabular}{|p{2cm}|p{1cm}|p{1.5cm}|p{8.5cm}|}\hline
Datum & Vers. & Kürzel & Änderung \\ \hline
19.04.2018 & 0.1 & FL & Erzeugung Dokument; Erzeugung Inhaltsverzeichnis; Erzeugung Versionierung;
Erzeugung 2.1 - 2.7.4 \\ \hline
19.04.2018 & 0.2 & FL & Korrekturen 2.6.1 - 2.6.9 u. 2.7.1 - 2.7.2 Titel\\ \hline
20.04.2018 & 0.2.1 & FL & Erzeugung 2.7.1.1 - 2.7.1.4; Korrektur Riemannsche Untersumme; Erzeugung Literaturverzeichnis \\ \hline
\end{tabular}

\newpage
 
\section{\underline{Integralberechnung}}
\subsection{Unbestimmtes Integral}
\begin{equation}
\int f(x) dx = F(x) + C = [F(x)]\qquad, C\in\R \label{eq:def_noBorder}
\end{equation}

\subsection{Bestimmtes Integral}
\begin{equation}
\int_a^b f(x) dx = F(b) - F(a) \label{eq:def_border}
\end{equation}

\subsection{Partielle Integration}
Entspricht der "Produktregel" der Differentialrechnung.
\begin{equation}
\int_{\colBord{a}}^{\colBord{b}} f'(x) g(x) dx = f(x) g(x) \colBord{\Big|_a^b} - \int_{\colBord{a}}^{\colBord{b}} f(x) g'(x) dx \label{eq:rule_partInt}
\end{equation}
Bietet sich zum Beispiel bei Produkten aus x-Potenz mit e-Funktionen, log, sin oder cos an.

\subsection{Integration durch Substitution}
Entspricht der "Kettenregel" der Differentialrechnung.
\begin{equation}
\int_{\colBord{a}}^{\colBord{b}} f(g(x))g'(x) dx = \int_{\colBord{g(a)}}^{\colBord{g(b)}} f(y) dy \qquad (setze \quad y = g(x) \label{eq:rule_subs}
\end{equation}

\subsubsection{Spezialfall}
\begin{equation}
\int \frac{f'(x)}{f(x)} dx = ln(|f(x)|) + C \qquad ,C\in\R \label{eq:rule_spec}
\end{equation}

\newpage

\subsection{Gerade/Ungerade Funktionen}
\begin{align}
\int_{-a}^a f(x) = 
\begin{cases}
2 \int_0^a f(x) dx &,\; f\; gerade\\
0 &,\; f\; ungerade\\
\end{cases} \label{eq:evenodd}
\end{align}
\begin{align*}
\text{f gerade, falls }f(-x) &= f(x) \qquad &(z.B.: cos(x), x^2)\\
\text{f ungerade, falls }f(-x) &= -f(x) \qquad &(z.B.: sin(x), x^3)\\
\end{align*}

\subsection{Beispiele}
\subsubsection{$\int_1^2 \frac{ln(t)}{t} dt$}
\begin{align}
\colBord{s }\; &\colBord{= ln(t) \Rightarrow \frac{ds}{dt} \Rightarrow ds} \;\colBord{=} \colBord{\frac{1}{t}dt} \nonumber\\
\colBord{Grenzen: t } &\colBord{=} \;\colBord{1, t=2 \rightarrow s = ln(1), s = ln(2)} \nonumber \\
\int_1^2 \frac{ln(t)}{t} dt &= \int_1^2 ln(t) \frac{1}{t} dt = \int_{ln(1)}^{ln(2)} s ds 
= \frac{1}{2} s^2 \colBord{\Big|_{s = ln(1) = 0}^{ln(2)}} = \frac{1}{2}(ln(2))^2 
\end{align}
Bauart des Integrals: $\int h(t)h'(t)dt$

\subsubsection{$\int_0^{\frac{1}{2}} tan(t) dt$}
\begin{align}
\colBord{s = cos(t) \rightarrow \frac{ds}{dt}} \; &\colBord{=} \; \colBord{-sin(t) \rightarrow ds = -sin(t) dt} \nonumber \\
\colBord{Grenzen: t = 0, t } \; &\colBord{=} \; \colBord{1 \rightarrow s = cos(0), s = cos(\frac{1}{2}} \nonumber \\
\int_0^{\frac{1}{2}} tan(t) dt = \int_{0}{\frac{1}{2}} \frac{sin(t)}{cos(t)} dt &= \int_0 ^{\frac{1}{2}} \frac{1}{cos(t)} sin(t) dt 
= - \int_{cos(0)}^{\frac{1}{2}} \frac{1}{s} ds \nonumber \\
= -ln(s)\colBord{\Big|_{s = cos(0)}^{cos(\frac{1}{2})}} &= -ln(cos(\frac{1}{2})) + ln(\usGreen{=1}{cos(0)})
\end{align}
Bauart des Integrals: $\int \frac{1}{h(t)}h'(t)dt$

\subsubsection{$\int 4xe^{-x} dx$}
\begin{align}
\int 4xe^{-x} dx &= 4 \int \usGreen{u}{x} \usGreen{v'}{e^{-x}} dx 
\overset{\colGreen{p.I.}}{=} \usGreen{u}{x} \usGreen{v}{-e^{-x}} 
-4\int \usGreen{u'}{1} \usGreen{v}{-e^{-x}} dx \nonumber\\
&= -4xe^{-x} -4e^{-x} + C 
\qquad C\in\R \label{eq:ex1}
\end{align}

\subsubsection{$\int_0^{\pi}(x+3)cos(2x) dx$)}
\begin{align}
\int_0^{\pi} \usGreen{u}{(x+3)} \usGreen{v'}{cos(2x)}dx 
&\overset{\colGreen{p.I.}}{=} \underbrace{\usGreen{u}{(x+3)} * \frac{1}{2}\usGreen{v}{sin(2x)}\Big|_0^{\pi}}_{\colBord{=0}}
- \frac{1}{2} \int_0^{\pi} \usGreen{u'}{1} * \usGreen{v}{sin(2x)} \nonumber \\
&\;=\frac{1}{4}\; cos(2x)\Big|_0^{\pi} = 0 \label{eq:ex2}
\end{align}

\subsubsection{$\int cos^2(x)dx$}
\begin{align}
\int cos^2(x)\;dx &= \int \usGreen{u}{cos(x)} * \usGreen{v'}{cos(x)}\;dx \overset{\colGreen{p.I.}}{=}
\usGreen{u}{cos(x)} * \usGreen{v}{sin(x)} - \int \usGreen{u'}{(-sin(x))}*\usGreen{v}{sin(x)}\;dx \nonumber\\
&=cos(x)sin(x) + \int \underbrace{sin^2(x)}_{\colBord{= 1-cos^2(x)}}dx = cos(x)sin(x) + \int 1dx - \int cos^2(x)\; dx \nonumber\\
&\Rightarrow 2\int cos^2(x)\;dx = cos(x)sin(x) + x + \tilde{c} \qquad ,\tilde{c}\in\R\nonumber\\
&\Rightarrow \int cos^2(x)\;dx = \frac{cos(x)sin(x)+x}{2}+C \qquad, C\in\R \label{eq:ex3}
\end{align}

\subsubsection{$\int_0^1 x arctan(x) dx$}
\begin{align}
\int_0^1 \usGreen{u'}{x} \usGreen{v}{arctan(x)} dx &\overset{\colGreen{p.I.}}{=} 
\frac{1}{2} * \usGreen{u}{x^2} \usGreen{v}{arctan(x)} \Big|_0^1 - 
\int_0^1 \usGreen{u}{\frac{1}{2} x^2} \usGreen{v'}{\frac{1}{1+x^2}}dx \nonumber\\
&\;= \frac{\pi}{8} - \frac{1}{2} \int_0^1 \frac{1}{1+x^2} dx 
= \frac{\pi}{8} - \frac{1}{2} \int_0^1 1 dx + \int_0^1 \frac{1}{1+x^2} dx \nonumber\\
&\;= \frac{\pi}{8} - \frac{1}{2}* (x\Big|_0^1) + \frac{1}{2} actan(x)\Big|_0^1 
= \frac{\pi}{8} - \frac{1}{2} * \frac{\pi}{8} \nonumber\\
&\;= \frac{\pi}{4} - \frac{1}{2} \label{eq:ex4}
\end{align}

\subsubsection{$\int \frac{3x^2+2}{x^3+2x+1}dx$}
\begin{align}
\int \frac{3x^2+2}{x^3+2x+1}dx = ln(|x^2+2x-1|) + C \qquad ,C\in\R\label{eq:ex5}
\end{align}
Da das Integral über einen Ausdruck der Form $\frac{f'(x)}{f(x)}$ gebildet wird gilt \eqref{eq:rule_spec}. Alternativ Substitution:
\begin{align*}
y = x^3+2x-1 \Rightarrow dy = (3x^2+2) dx \Rightarrow \int \frac{1}{y}dy
\end{align*}

\subsubsection{$\int cos(x) e^{3sin(x)}dx$}
\begin{align}
\colBord{y = 3sin(x) \Rightarrow dy}\; &\;\;\colBord{=} \; \colBord{3cos(x) dx} \nonumber\\
\int cos(x) e^{3sin(x)}dx &\overset{\colGreen{subs.}}{=} \frac{1}{3} \int e^y dy \nonumber\\
&\;= \frac{1}{3} e^y + C \usGreen{R.s.}{=} \frac{1}{3}e^{3sin(x)} + C \qquad , \; C\in\R
\end{align}

\subsubsection{$\int \frac{1}{(2+x) \sqrt{1+x}}dx$}
\begin{align}
\colBord{x = 2sin(y) \Rightarrow dx} &\;\;\colBord{=} \colBord{2cos(y) dy} \nonumber\\
\int \frac{1}{(2+x) \sqrt{1+x}}dx &\usGreen{subs.}{=} 
\int \frac{1}{(y^2 + 1) \colRed{\cancel{y}}} 2\colRed{\cancel{y}} dy 
= 2\int \frac{1}{1+y^2}\;dy \nonumber\\
&\;= 2*arctan(y) + C = 2* arctan(\sqrt{1+x} + C ,\;C\in\R
\end{align}

\subsubsection{$\int \sqrt{4+x^2} \; dx$}
\begin{align}
\colBord{x = 2 sin(y) \Rightarrow dx} &\;\;\colBord{=} \colBord{\;2cos(y)dy} \nonumber\\
\int \sqrt{4+x^2} \; dx &\overset{\colGreen{subs.}}{=}
\int \sqrt{4-4sin^2(y)} * 2cos(y) \; dy \nonumber\\
&\;= \int \sqrt{4cos^2(y)} 2cos(y) \; dy = 4\int cos^2(y) \; dy \nonumber\\
&\overset{\colGreen{p.I.}}{=} 4*(\usGreen{u}{cos(y)} * \usGreen{v}{sin(y)} - \int \usGreen{u'}{(-sin(y))}*\usGreen{v}{sin(y)}\;dy) \nonumber\\
&\;\Rightarrow 4\int cos^2(y) dy = cos(y)sin(y) + \int 1dy - \int cos^2(y)\; dy \nonumber\\
&\;\Rightarrow 5\int cos^2(y)\;dy = cos(y)sin(y) + y + \tilde{c} \qquad ,\tilde{c}\in\R\nonumber\\
&\;\Rightarrow \int cos^2(y)\;dy = \frac{cos(y)sin(y)+x}{5}+C \qquad, C\in\R
\end{align}

\subsubsection{$\int_{-3}^3 1+e^{x^2} *(sin(x)^3\;dx$}
\begin{align}
\int_{-3}^3 1+e^{x^2} *(sin(x)^3\;dx &= \underbrace{\int_{-3}^3 1 dx}_{\colBord{= 6}} 
+ \underbrace{\int_{-3}^{3} e^{x^2} * (sin(x))^2 dx}_{\colBord{ = 0 \text{ da der Integrand 
eine ungerade Funktion ist}}} \nonumber\\
&= 6
\end{align}
Untersucht man den Integranden, stellt man seine Ungeradheit leicht fest.
\begin{align*}
sin(x) &\Rightarrow ungerade \\
sin^2(x) &\Rightarrow gerade \\
sin^3(x) &\Rightarrow ungerade
\end{align*} 
Funktionen verhalten sich im Bezug auf Ungeradheit zur Multiplikation ähnlich wie Vorzeichen.
($(-)*(-) = (+);\; (-) * (+) = (-)$)
Untersucht man weiterhin die e-Funktion stellt man fest, dass sie gerade ist:
\begin{align*}
f(x) = e^{x^2} = e^{(-x)^2} = f(-x) \Rightarrow gerade \; Funktion
\end{align*}
Da also eine gerade mit einer ungeraden Funktion multipliziert wird ist das Ergebnis wiederrum ungerade.

\subsection{Allgemeines zur Integration}
\subsubsection{Riemann Integrierbarkeit}
$f:[a,b] \rightarrow \R$ stetig bzw. monoton \newline
$\Rightarrow$ f ist R-integrierbar.

\subsubsubsection{Riemannsches Unterintegral}
\begin{equation}
\int_{a}^{\bar{b}} f(x) dx = \sup\{U_f(Z): \; \text{Z Zerlegung von }[a,b]\}
\end{equation}


\subsubsubsection{Riemannsches Oberintegral}
\begin{equation}
\int_{\bar{a}}^{b} f(x) dx = \inf\{O_f(Z): \; \text{Z Zerlegung von }[a,b]\}
\end{equation}

$\rightarrow \text{f heißt Riemann-integrierbar über }[a,b]$, falls
\begin{equation}
\int_{\bar{a}}^{b} f(x) dx = \int_a^{\bar{b}} f(x) dx
\end{equation}
\newline
In diesem Fall heißt der Wert das Riemannn-Integral und wird mit $\int_a^b f(x)dx$ bezeichnet.

\subsubsubsection{Eigenschaften}
\begin{description}
\item[a)]
Falls $a<b$ setzen wir:
\begin{align}
\int_b^a f(x) dx &= -\int_a^bf(x)dx \nonumber \\
\int_a^a f(x) dx &= 0
\end{align}
\item[b)]
f, g seien R-integrierbar, $\lambda , \mu \in \R \rightarrow \lambda f + \mu g$ ist R-integrierbar (Vektorraumeigenschaft).
\begin{equation}
\int_a^b \lambda f + \mu g)(x)dx = \lambda \int_a^b f(x) dx + \mu \int_a^b g(x) dx
\end{equation}
\item[c)]
$a<C<b$, f ist R-integrierbar.
\begin{equation}
\int_a^b f(x) dx = \int_a^C f(x) dx + \int_C^b f(x) dx
\end{equation}
\item[d)]
\begin{align}
f(x) \ge 0 &\Rightarrow \int_a^b f(x) dx \ge 0 \nonumber \\
f(x) \ge g(x) &\Rightarrow \int_a^b f(x) dx \ge \int_a^b g(x)dx
\end{align}
\item[e)]
\begin{equation}
\text{Sind $f$ und $g$ R-integrierbar ist auch $f*g$ R-integrierbar.}
\end{equation}
\item[f)]
\begin{align}
g(x) \ge C > 0 \Rightarrow \frac{f}{g} \text{ ist R-integrierbar.}
\end{align}
\item[g)]
\begin{equation}
\text{Ist $f$ R-integrierbar dann ist auch } |f| \text{ R-integrierbar.}
\end{equation} 
\item[h)]
\begin{equation}
(b-a) \inf_{x\in[a,b]}{f(x)} \le \int_a^b f(x) dx \le (b-a) \sup_{x\in [a,b]}{f(x)}
\end{equation}
\end{description}

\subsubsubsection{Kriterien zur Riemann-Integrierbarkeit}

\begin{description}
\item[a)]
$f$ monoton $\Rightarrow f$ R-integrierbar.
\item[b)]
$f$ stetig $\Rightarrow f$ R-integrierbar
\newline
\glqq Satz: Jede stetige Funktion $f:k \rightarrow \R$ auf einer kompakten Menge k, d.h. für $k<\R^d$ abgeschlossen und beschränkt, ist dort gleichmäßig stetig und damit R-integrierbar.\grqq \cite{HM12}
Beispiel für k: $k:[a,b]$
\item[c)]
Kriterium: Jede Funktion deren Unstetigkeitsstellen eine Nullmenge bilden (z.B. abzählbare Mengen) sind R-integrierbar.
\glqq Satz: Eine Funktion $f:[a,b]\rightarrow \R$ ist genau dann R-integrierbar, wenn $f$ beschränkt ist und die Menge der Unstetigkeitsstellen eine Nullmenge ist. \grqq \cite{HM12}
Die Konsequenz daraus lautet, dass jede stetige Funktion mit endlich vielen Sprungstellen R-integrierbar ist. \citeVgl{HM12}
\item[d)]
\glqq Satz: Sei $f:[a,b] \rightarrow \R$ beschränkt. Dann ist $f$ R-integrierbar genau dann, wenn es zu jedem $\varepsilon > 0$ eine Partition $Z$ gibt, 
so dass
$O_f(Z)  U_f(Z) < \varepsilon$. \grqq \cite{HM12}
\newline
Anmerkung: \glqq In der Mengenlehre ist eine Partition (auch Zerlegung oder Klasseneinteilung) einer Menge M eine Menge P, deren Elemente nichtleere Teilmengen von M sind, sodass jedes Element von M in genau einem Element von P enthalten ist. Anders gesagt: Eine Partition einer Menge ist eine Zerlegung dieser Menge in nichtleere paarweise disjunkte Teilmengen.\grqq  \cite{wiki}

\end{description}

\subsubsection{MWS der Integralrechnung}
$f:[a,b]\rightarrow\R$ stetig, dann $\exists \; \xi \in[a,b]$ mit $\int_a^b f(x)dx = f(\xi)(b-a)$.

\subsubsection{Hauptsatz der Differential- und Integralrechnung}
$f:[a,b]\rightarrow\R$ stetig, dann ist $F(x) = \int_a^x f(t)dt$ diffbar und $F'(x) = f(x)$.

\subsubsection{Anwendungen}
\begin{description}
\item[1)]
\begin{align}
\lim_{x\to 0} \frac{1}{x} \int_0^x e^{-cos(y^{17})} dy = \lim_{x\to 0}  \frac{\overbrace{\int_0^x e^{-cos(y^{17})}}^{\colBord{\rightarrow 0}}}{\underbrace{x}_{\colBord{\rightarrow 0}}} \nonumber\\
\overset{\colBord{"\frac{0}{0}" \rightarrow L.H.}}{=}
\lim_{x \to 0} \frac{e^{-cos(x^{17}}}{1} = e^{-1} = \frac{1}{e}
\end{align}
\item[2)]
\begin{align}
\lim_{x \to \infty} xe^{x^2} \int_0^x e^{y^2}dy 
&= \lim_{x \to \infty} \frac{\overbrace{\int_0^x e^{y^2}dy}^{\colBord{\rightarrow \infty}}}
{\underbrace{\frac{1}{x} e^{x^2}}_{\colBord{\rightarrow \infty}}} \nonumber\\
\overset{\colBord{"\frac{\infty}{\infty}" \rightarrow L.H.}}{=}
\lim_{x \to \infty} \frac{e^{x^2}}{-\frac{1}{x^2}* \colRed{\cancel{e^{x^2}}}
+ \frac{1}{x} 2x \colRed{\cancel{e^{x^2}}}} 
&= \lim_{x \to \infty} \frac{1}{- \frac{1}{x^2} +2} = \frac{1}{2}
\end{align}
\end{description}

\newpage
\section{\underline{Anhänge}}
\subsection{Formelverzeichnis}


\bibliography{lit}

\end{document}
